%\title{Math 453 HW 1}
\documentclass[addpoints]{exam}

\usepackage{amsmath,enumitem,wrapfig}
\usepackage{tikz}

\newcommand{\StudentName}{Student Name}
\newcommand{\AssignmentName}{HW 1}

\pagestyle{headandfoot}
\runningheadrule
\firstpageheadrule
\firstpageheader{Math 453}{\StudentName}{\AssignmentName}
\runningheader{Math 453}{\StudentName}{\AssignmentName}
\firstpagefooter{}{}{}
\runningfooter{}{}{}

\printanswers

\begin{document}


Organize your work and show any work that you want credit for. Use full sentences where possible.

\begin{questions}

\question \textbf{M1}
\begin{parts}
\part
Consider the arithmetic computation below.
\begin{align}
3+4[5-12]-6(3) +(4+0)
&= 3+4[5-12]-6(3) +4\\
&= 4[5-12]-6(3) +4+3\\
&= 20-48-18 +4+3 \\
&=-39.\notag
\end{align}
For each of the steps (1), (2), and (3) identify which of the Axioms of Integer Arithmetic are used in the simplification step.

\begin{solution}
Awesome answer.
\end{solution}

\part Create and simplify an expression that uses associativity of addition, multiplicative identity, and the distributive law.
\end{parts}

\question \textbf{M2}
For each statement below determine whether each statement is correct for integers $a$, $b$, and $c$. If the statement is correct, then prove it. If the statement is incorrect, then modify it so that it is correct. Be sure to state which Order Axiom(s) you have applied.
\begin{parts}
\part If $a<b$, then $c\cdot a< c\cdot b$.

\part If $a<b$, then $a+c<b+c$.

\part If $a<b$, $b<c$, and $c<d$, then $a<d$.

\part If $a\not > b$ and $a\not< b$, then $a=b$.
\end{parts}

\question \textbf{M3}
\begin{parts}
\part
Find the flaw in the following argument.
\begin{quote}
To solve $x(x+4)=x(2x-8)$ we divide both sides by $x$ (or apply Theorem 1.11) to get $x+4=2x-8$. Subtract $(x-8)$ from both sides to obtain $12=x$, so the solution is $x=12$. 
\end{quote}

\part 
Find the flaw in the following argument.
\begin{quote}
To solve $x(x-4)=12$ we factor the left-hand side and set the factors equal to zero $x=0$ and $x-4=0$ and conclude that $x=0,4$.  
\end{quote}

\end{parts}

\question \textbf{M4}
\begin{parts}
\part Work Exercise 1 from Investigation 1 (uniqueness of additive inverses).

\part Work Exercise 2 from Investigation 1 (additive cancellation).

\end{parts}





\end{questions}
\end{document}